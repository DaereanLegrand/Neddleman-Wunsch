\documentclass{article}
\usepackage[spanish]{babel}
\usepackage{hyperref}
\usepackage{graphicx}

\hypersetup{
    colorlinks=true,
    linkcolor=blue,
    filecolor=blue,      
    urlcolor=blue,
    pdftitle={Overleaf Example},
    pdfpagemode=FullScreen,
    }

\title{Laboratorio 01a: Neddleman-Wunsch}
\author{Frank Roger Salas Ticona}
\date{\today}

\begin{document}
\maketitle

\section{Introducción}
El algoritmo de Needleman-Wunsch es un método fundamental en bioinformática para alinear secuencias biológicas, como proteínas o ADN. Este trabajo presenta la implementación en C++ de este algoritmo, su evaluación en diferentes conjuntos de datos y un análisis de los resultados obtenidos. Se explorarán además algunas técnicas de visualización y se discutirán las mejoras propuestas en la literatura.

\section{Implementación}
\href{https://github.com/DaereanLegrand/Neddleman-Wunsch.git}{link}

\section{Resultados}
\subsection{Experimentos}

\begin{table}[ht]
    \centering
    \begin{tabular}{|l|c|c|c|}
        \hline
        Par de secuencias & Tamaño (aa) & Tiempo (s) & Soluciones óptimas \\
        \hline
    \end{tabular}
    \caption{Resumen de los resultados obtenidos.}
    \label{tab:resultados}
\end{table}

\subsection{Criterio de menos rupturas}

\subsection{Visualización}
\begin{figure}[ht]
    \centering
    %\includegraphics[width=0.8\textwidth]{matriz_puntos.png}
    \caption{Ejemplo de matriz de puntos para el alineamiento de dos secuencias.}
    \label{fig:matriz_puntos}
\end{figure}

\section{Análisis y Discusión}

\section{Conclusiones}

\bibliographystyle{plain}
\bibliography{bibliografia}

\end{document}
